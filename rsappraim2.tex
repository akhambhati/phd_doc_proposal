\subsection{Aim 2}
In this section I extend the notion of an epileptic network more broadly to the inter-ictal period and ask: can topological abnormalities in the network that present during seizures be identified pre-ictally? I \textbf{hypothesize} that fundamental abnormalities in the functional wiring of epileptic networks can map seizure-generating areas during inter-ictal periods.

\subsubsection{Justification}
This aim takes an incremental step towards translating network-based approaches for objectively identifying topological abnormalities to practical clinical applications. The methodology developed here can be applied to ask a variety of questions regarding inter-ictal dynamics such as why sub-clinical events begin like seizures but don't evolve.

\subsubsection{Research Design}
\textbf{Association Metrics:}
~\\
~\\
\textbf{Null Connectivity:}
~\\
~\\
\textbf{Dynamic Networks:}
~\\
~\\
\textbf{Network Configuration:}
~\\
~\\
\textbf{Communities and Sub-Networks:}
~\\
~\\
\textbf{Global and Local Architecture:}


