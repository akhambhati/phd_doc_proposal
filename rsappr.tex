\section{Approach}
This thesis will answer several important questions, in the engineering, science and clinical realms, with regards to diagnosing functional architecture and identifying surgical targets in localization-related epileptic networks.

\bflin{Aim 1:} \textbf{Identify topological abnormalities in the epileptic network as seizures begin and evolve.}
I ask the basic science question: how is the epileptic network organized in such a way that enables seizure generation, propagation and termination? I will examine and identify specific functional connections associated with temporal seizure dynamics. I \textit{hypothesize} that network-based algorithms can characterize epileptic network functional architecture and dissociate an abnormal seizure-generating network from a surrounding epileptogenic network.
~\\
~\\
\textbf{\textit{Justification:}
This aim is important to the proposed project because it will isolate mechanistic roles for clinically-determined seizure onset areas and their interactions with adjacent cortex as the epileptic network undergoes dynamic functional alterations through seizures. This study expands basic science understanding of well-known temporal phenomena in seizures to the lesser-studied spatial domain.
~\\
~\\
\textbf{\textit{Collaborations:}
For guidance in engineering methodology I will leverage collaborations with Danielle Bassett, an expert in complex networks at the University of Pennsylvania who studies structural and functional brain networks, and Gershon Buchsbaum, an expert in signal processing at the University of Pennsylvania who can advise on signal dynamics. For guidance in clinical neuroscience I will collaborate with Timothy Lucas, a neurosurgeon at the University of Pennsylvania and expert in targeted surgery and neuromodulation.
~\\
~\\
\textbf{\textit{Patient Data:}
For guidance in engineering methodology I will leverage collaborations with Danielle Bassett, an expert in complex networks at the University of Pennsylvania who studies structural and functional brain networks, and Gershon Buchsbaum, an expert in signal processing at the University of Pennsylvania who can advise on signal dynamics. For guidance in clinical neuroscience I will collaborate with Timothy Lucas, a neurosurgeon at the University of Pennsylvania and expert in targeted surgery and neuromodulation.



