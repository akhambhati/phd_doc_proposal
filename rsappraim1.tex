\subsection{Aim 1}
I ask the basic science question: how is the epileptic network organized in such a way that enables seizure generation, propagation and termination? I will examine and identify specific functional connections associated with temporal seizure dynamics. I \textbf{hypothesize} that network-based algorithms can dissociate an abnormal seizure-generating network from a surrounding epileptogenic network.

\subsubsection{Justification}
This aim is important to the proposed project because it will isolate mechanistic roles for clinically-determined seizure onset areas and their interactions with adjacent cortex as the epileptic network undergoes dynamic functional alterations through seizures. This study expands basic science understanding of well-known temporal phenomena in seizures to the lesser-studied spatial domain.

\subsubsection{Research Design}
\begin{enumerate}
    \item Seizure segments (EEC), pre-seizure segments
\end{enumerate}
\textbf{Association Metrics:}
~\\
~\\
\textbf{Null Connectivity:}
~\\
~\\
\textbf{Dynamic Networks:}
~\\
~\\
\textbf{Network Configuration:}
~\\
~\\
\textbf{Communities and Sub-Networks:}
~\\
~\\
\textbf{Global and Local Architecture:}


