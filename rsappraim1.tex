\subsection{Aim 1}
\label{rsappr:aim1}
I ask the basic science question: how is the epileptic network organized in such a way that enables seizure generation, propagation and termination? I will examine and identify specific functional connections associated with temporal seizure dynamics. I \textbf{hypothesize} that network-based algorithms can dissociate an abnormal seizure-generating network from a surrounding epileptogenic network.

\subsubsection{Justification}
This aim is important to the proposed project because it will isolate mechanistic roles for clinically-determined seizure onset areas and their interactions with adjacent cortex as the epileptic network undergoes dynamic functional alterations through seizures. This study expands basic science understanding of well-known temporal phenomena in seizures to the lesser-studied spatial domain.

\subsubsection{Research Design}
To address this aim partial and secondarily generalized seizures from the patient population (see \ref{subsec:patients}) will be extracted using the IEEG Portal and MATLAB IEEG Toolbox. Data processing thereafter will utilize Python and its public toolboxes.  The seizure epoch will range the clinically-defined earliest electrographic change (EEC), believed to be the earliest dynamical change from background signal characteristics \cite{litt2001epileptic}, to the end of ictal flickering (prior to post-ictal quieting). The pre-seizure epoch will be used as a control signal and will span a duration equal to its associated seizure epoch and end at the EEC. Each epoch contains dynamics that transition between seizure initiation, spread or propagation, and termination. Described herein are steps I will pursue to study the functional mechanisms and connections that drive such dynamics.
~\\
~\\
\textbf{Association Metrics:}
To study spatial interactions within a network a common practice is to define the notion of a \textit{node} and a \textit{connection} \cite{bullmore2011brain}. The node represents the most basic functioning unit of a network and is dependent on the signal scale, whether it be at the scope of single neurons or their local populations. Meso-scale ECoG recordings limit the notion of a node to a single electrode contact (see \ref{subsec:elecdata}) that senses surface-level, local neuron populations \cite{buzsaki2012origin}. Connectivity, or interactions between neuron populations over sensors, can be quantified by an association metric.

An association metric, whether a linear or non-linear measurement, is often justified based on a data-driven feature of what it means for nodes to be interacting or connected. Meso-scale temporal dynamics are often characterized by their rhythmic or oscillatory behavior over a range of clinically-relevant frequency bands and are known to play a major role in normal cognition \cite{buzsaki2006rhythms} and in diseases such as epilepsy \cite{uhlhaas2006neural, jiruska2012synchronization}. Interactions between nodes of this scale primarily fall into recruitment and isolation through rhythmic synchrony and desynchrony. Formally, network nodes that are connected (disconnected) are synchronized (desynchronized). (De)synchronization amongst network nodes can be described through linear association metrics such as cross-correlation among others \cite{pereda2005nonlinear}. I will use normalized cross-correlation (\ref{eqn:xcorr}) to relate nodal dynamics based on prior work showing the metric's neurophysiologic interpretability and efficiency in uncovering phase relationships \cite{schiff2005neuronal, kramer2010coalescence}. Normalized cross-correlation $\bs{\rho}$ between the signals of a pair of nodes $x(t)$ and $y(t)$ is defined by
\begin{eqnarray}
\label{eqn:xcorr}
    \bs{\rho}_{xy} = \underset{\tau}{\operatorname{argmax}}\:{\abs{\mathrm{E}[(x(t) - \mu_{x})(y(t+\tau) - \mu_{y})]}}
\end{eqnarray}
where $\tau$ is the time-shift between the signals and $\mu$ is the mean of the signal. $\bs{\rho}$ ranges between 0 and 1 corresponding to a connection weight between the nodes.

The collection of all possible connection weights are stored in a symmetric, $N \times N$ association matrix $\mb{A}$ where $N$ is the number of nodes in the network with $\mfrac{N(N-1)}{2}$ unique connections (for simplicity $n=\mfrac{N(N-1)}{2}$). The fundamental goal of this aim is to identify and understand the dynamic topological relationships of these connections during seizures.
~\\
~\\
\textbf{Null Connectivity:}
A novel aspect of this work is studying weighted connectivity between nodes, unlike other studies which binarize connectivity based on single threshold values. Retaining connection strengths enables more discriminative power in assigning functional roles to specific connections \cite{bullmore2011brain}. This bonus does come with drawbacks that must be addressed to make meaning of the large number of connections in the network.

Null connectivity models that inform which connections are statistically probable through multiple comparisons testing are necessary here \cite{bassett2013robust}. In this work the goal is to determine which associations appear to have high cross-correlation values by chance. I will first create a null distribution of $k$ surrogate association matrices $\widetilde{\mb{A}}_k$ based on association metric $\bs{\rho}$ through time-series bootstrapping methods such as random permutation, Fourier phase randomization, or amplitude-adjusted Fourier phase randomization \cite{bassett2013robust}. Fourier phase randomization destroys correlative phase relationships between nodes through a pipeline that involves (1) forward Fourier transform, (2) randomly reassigning phase coefficients, (3) and inverse Fourier transform.

Second, based on $\mb{A}$ and $\widetilde{\mb{A}}$ I will calculate and store a two-tailed p-value for each possible connection in a symmetric, $N \times N$ connection probability matrix $\mb{P}$. To retain significant connections over multiple hypotheses testing we will apply a false discovery rate method on $\mb{P}$ and zero-out corresponding non-significant connection entries in $\mb{A}$ \cite{benjamini2001control}. Through this process the association matrix becomes the network adjacency matrix $\widehat{\mb{A}}$, which is versatile because it contains significant, upper-tailed strong and lower-tailed weak connections.
~\\
~\\
\textbf{Dynamic Connectivity:}
I will lay the groundwork for studying the epileptic network in the context that functional connectivity changes with time as different neural populations are engaged as shown during cognitive tasks in healthy subjects \cite{bassett2011dynamic}. Such a perspective is novel in epilepsy and has not yet been examined in networks consisting of meso-scale architecture.

Clinicians and researchers routinely describe frequency-dependent dynamics on a per sensor basis as seizures evolve \cite{franaszczuk1998timefrequency, tzallas2009epileptic}, but there is little evidence as to how such dynamics behave spatially within the network. Under the assumption that band-specific neural activity may synchronize regions in the epileptic network, this recruitment during key phases of seizure evolution can provide critical insight to abnormal network structures. I will extend the definition of an epileptic network by re-defining the adjacency matrix as $\widehat{\mb{A}}(f)$ where $f$ is one of four clinically-relevant frequency bands (4-8, 8-13, 13-30, 30-100 Hz).

In order to consider time-dependent connectivity I will divide signals into time-windows of a particular duration. Adding time-dependency modifies the adjacency matrix representation to $\widehat{\mb{A}}(t)$ where $t$ is one of $\mb{T}$ time-window. Studies suggest that one second of ECoG signal is weakly stationary \cite{kramer2010coalescence}. This approach will need to factor the number of cycles of frequency-dependent rhythmic activity within the chosen time-window in order to gather an accurate estimate of correlated activity.

Specifically in this study I will model dynamic functional connectivity within the seizure epoch and pre-seizure, control epoch and generate two different sets of time-frequency dependent adjacency matrices $\widehat{\mb{A}}_{Sz}(f,t)$ and $\widehat{\mb{A}}_{preSz}(f,t)$, respectively.
~\\
~\\
\textbf{Network States:}
The functional connectivity within a given time-window is a single network configuration snapshot. Mathematically, I will represent this snapshot by a configuration vector $\vec{v}$ with $n$ elements representing number of connections, which are obtained by unraveling the upper-triangle of a symmetric adjacency matrix. I extend this notion of a configuration vector to frequency-band and time-window dependency as $\vec{v}(f,t)$.

One way to study dynamic configuration is to quantify how network structure alters through a variety of states over time. I will describe epileptic network states as seizures evolve based on grossly stereotyped connectivity. Given the dynamic configuration vector $\vec{v}(f=f_1,t)$, where $f_{1}$ is a fixed frequency-band, I will compare gross network structure between $t=t_1$ and $t=t_2$ for all possible pairs of time-windows $t$ using association metrics such as Pearson correlation or Jaccard index. The resulting $T \times T$ comparisons form a fully-connected network-state adjacency matrix $\mb{S}$, where each entry describes configuration similarity between time-windows.

I will apply clustering algorithms on $\mb{S}$ to assign time-windows with similar network configurations into a cluster or network state. This approach is valuable to this study because (1) it will identify transitions or reconfigurations in network structure during seizures that can be related to neurophysiology and (2) it enables detailed analysis and localization into how and where connectivity is different in different seizure stages.
~\\
~\\
\textbf{Community Detection:}
The network-state adjacency matrix $\mb{S}$ is a fully-connected graph where each node is a time-window and edges between nodes are weighted configuration similarities. I will use an unsupervised network clustering approach known as community detection on $\mb{S}$ to uncover network states in the dynamic epileptic network $\widehat{\mb{A}}(f=f_1,t)$.

Community-detection is used to study meaningful group structure in complex networks by clustering nodes into `communities'. A community is a set of nodes that are connected among one another more densely than they are to nodes in other communities. A popular way to identify community structure is to optimize a quality function known as modularity $Q$, which has been shown to extract meaningful functional components of networks by comparing the real network to a null model \cite{newman2004finding, newman2006modularity, porter2009communities, fortunato2010community}.

I will detect communities (here, equivalent to a network state) in $\mb{S}$ by optimizing modularity
\begin{eqnarray}
\label{eqn:modopt}
	Q = \sum_{ij} [S_{ij} - \gamma P_{ij}] \delta(g_{i},g_{j})
\end{eqnarray}
where node $i$ is assigned to community $g_{i}$, node $j$ is assigned to community $g_{j}$, the Kronecker delta $\delta(g_{i},g_{j})=1$ if $g_{i} = g_{j}$ and it equals $0$ otherwise, $\gamma$ is the \emph{structural resolution parameter}, and $P_{ij}$ is the expected weight of the edge connecting node $i$ to node $j$ under a specified null model. The choice $\gamma = 1$ is very common, but it is important to consider multiple values of $\gamma$ to examine groups at multiple scales. A commonly chosen null model is the Newman-Girvan null model \cite{porter2009communities, fortunato2010community, newman2004finding, newman2006modularity}.  Maximizing $Q$ partitions the network into communities such that the total within-community connection weight is maximum relative to the null.

Parsing time-dependent networks into different states using community detection is novel. While I introduce community detection as a tool for assigning network states, I note that it can be applied to dynamic adjacency matrices $\widehat{\mb{A}}_{Sz}(f,t)$ and $\widehat{\mb{A}}_{preSz}(f,t)$ to extract functional modules in the epileptic network.
~\\
~\\
\textbf{Global and Local Network Geometry:}
I will quantify and compare connectivity between epileptic network states in seizure and pre-seizure epochs using a variety of network metrics. To identify global structural differences in connectivity I will study network properties such as average connection strength and connectivity imbalance. The former will quantify whole-network connectedness to uncover the impact of seizure phases on network structure, and the latter will elucidate how epileptic networks support seizure evolution through push-pull control processes that regulate network synchronization \cite{he2014control}. These insights can also improve sensitivity in seizure detection tools for clinical applications.

To examine local epileptic network geometry I intend to track specific network connections through pre-seizure and seizure epochs and determine which connections demarcate specific desynchronous and synchronous zones within the network and guide seizure dynamics. First I will classify connection strengths as weak (desynchronized) or strong (synchronized) types based on their distribution percentile rank within each epoch. Clinical marking of where seizures begin will inform anatomical classification of connections as between nodes (1) inside, (2) outside, and (3) inside and outside the seizure onset zone. Thus I propose to study prevalence of connection type and connection location during detected epileptic network states. This comprehensive analysis can isolate how (1) seizure onset areas reconfigure as seizures are generated, (2) pathologic activity spreads outside seizure onset areas, and (3) the network recruits other nodes to terminate seizures.

\subsubsection{Preliminary Data}
Initial experiments conducted on seizure segments extracted from seven patients with partial seizures that secondarily generalize provide encouraging results and support this network-based approach in a broader patient pool. Using aforementioned dynamic network techniques I extract network states during pre-seizure and seizure epochs (Fig. \ref{fig:netstate}) and then analyze connection geometry within those states to identify which network regions are compromised (Fig. \ref{fig:netgeom}).

Preliminary network state analysis demonstrate that most seizures experience at least three dynamical phases (Fig. \ref{fig:netstate}B), which occupy the bulk of the seizure duration, and that these phases align with what clinicians consider seizure initiation, propagation and termination (Fig. \ref{fig:netstate}A). These results suggest dynamic network methods are capable of parsing seizure dynamics into meaningful states and can be applied broadly to study a variety of seizure morphologies.

\begin{figure}[t]
\centering
\includegraphics[width=0.75\textwidth]{figs/panelNetState.eps}
\caption[Dynamic epileptic network states]{\textbf{Distinct States of Dynamic Epileptic Networks.} (\textit{A}) Example assignments of sensors to network configuration communities (seizure states) demonstrating network reconfiguration. Assignments are overlaid on a subset of representative ECoG grids during an entire seizure. (\textit{B}) Number of communities in pre-seizure (gray) and seizure (blue) epochs; $t$-test ($N=25$, $t=0.50$, $p\approx0.62$).}
\label{fig:netstate}
\end{figure}

For each seizure I relate connection type and connection location to state dynamics (Fig. \ref{fig:netgeom}A). While these results align with analyses from other work \cite{schindler2006assessing, schindler2008evolving, kramer2010coalescence, jiruska2012synchronization}, my methods extract local network structure and show competition between globally desynchronized activity and locally synchronous activity as seizures evolve. Locations of this desynchronized and synchronized activity are related to physical space that delineate a potential seizure-generating region and surrounding epileptogenic region (Fig. \ref{fig:netgeom}B) and characterize how these regions interact.

\begin{figure}
\centering
\includegraphics[width=0.75\textwidth]{figs/panelNetGeom.eps}
\caption[Dynamic network geometry]{\textbf{Dynamics of Network Geometry.} (\textit{A}) Example of network geometry with preserved 2-D spatial relationships between nodes for categories of \emph{weak} ($<10\%$), \emph{medium} ($10$-$90\%$) and \emph{strong} ($>90\%$) connections; topographic map represents mean connection strength within a seizure community or pre-seizure epoch. The clinically-determined seizure onset sensors are shown in red. (\textit{B}) Population average balance index, $B$, for three connection locations (within seizure-onset zone, outside seizure-onset zone, and between seizure-onset zone and other sensors) across epochs and communities. (\textit{C}) Physical length of connections in a large $8x8$ grid, normalized to one for longest possible connection, averaged over connections and seizures for each configuration community. Error bars indicate standard error of the mean over seizures.}
\label{fig:netgeom}
\end{figure}
