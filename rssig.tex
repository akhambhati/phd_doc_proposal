\section*{Significance}
Epilepsy is a neurological disease of recurring seizures that affects an estimated 50 million people worldwide \cite{kwan2000early}. One-third of this population experiences seizures that are refractory to medication \cite{kwan2000early}. In $\sim$80\% of these patients seizures presumably originate from one or multiple focal cortical structures. These localization-related epilepsy patients undergo continuous and invasive monitoring of meso-scale brain activity through neural sensors recording the electrocorticogram (ECoG) to determine the seizure origin as a target for surgical resection, or more recently implantable neuromodulation devices. Seizure freedom is favorable in surgical cases where epileptic foci are localized to discrete lesions on brain MRI. However, only $\sim$40\% of non-lesional patients remain seizure free following resection \cite{french2007refractory}. The poor outcome associated with the invasiveness of these procedures has resulted in a paradigm shift from localizing epileptic `foci' towards mapping the epileptic network and identifying key regions for intervention \cite{spencer2002neural, kramer2012epilepsy, lehnertz2014evolving}.

The significance of the proposed thesis study is that it has the potential to improve outcome and to decrease risks associated with the current duration of invasive monitoring in patients undergoing surgery to treat localization-related refractory epilepsy. Epileptic foci localization has traditionally been addressed by studying emission rates of temporally defined electrophysiologic markers, often discarding spatial relationships between neural sensors \comment{[needs refs]}. While these approaches have yielded key insights to temporal scales of epilepsy \cite{litt2001epileptic}, until recently it was unclear how pathologic events are spatially coordinated given that the brain is highly inter-connected. A key turning point was the novel perspective that epilepsy is a disease of neuronal circuitry through which spreading pathologic activity disrupts neural function (i.e. rhythmic motor activity, altered cognition, or abnormal sensation) \cite{spencer2002neural}. Since, investigators have focused on the spatial distributions of epileptic activity using multiscale neural signals in ECoG and sub-millimeter $\mu$ECoG to more accurately localize where seizures start and how their pathologic activity affects distributed cortical structures \cite{worrell2008high-frequency, schevon2009spatial, stead2010microseizures, viventi2011flexible, feldt_muldoon2013spatially, weiss2013ictal}. More formally, investigators seek the functional organization of a complex epileptic network.


