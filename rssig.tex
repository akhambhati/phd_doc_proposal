\section{Significance}
Epilepsy is a neurological disease of recurring seizures that affects an estimated 50 million people worldwide \cite{kwan2000early}. One-third of this population experiences seizures that are refractory to medication \cite{kwan2000early}. In $\sim$28\% of refractory patients seizures presumably originate from one or multiple focal cortical structures in the neocortex. These neocortical epilepsy patients undergo continuous and invasive monitoring of meso-scale brain activity through neural sensors recording the electrocorticogram (ECoG) to determine the seizure origin as a target for surgical resection, or more recently implantable neuromodulation devices. Seizure freedom is favorable in surgical cases where epileptic foci are localized to discrete lesions on brain MRI. However, only $\sim$40\% of non-lesional patients remain seizure free following resection \cite{french2007refractory}. The poor outcome associated with the invasiveness of these procedures has resulted in a paradigm shift from localizing epileptic `foci' towards mapping the epileptic network and identifying key regions for intervention \cite{spencer2002neural, kramer2012epilepsy, lehnertz2014evolving}.

A \textbf{central challenge} in translating the notion of an epileptic network to guide clinical intervention is understanding the functional network components that drive seizure generation, propagation and self-termination. Many studies describe dynamic topological trends in global epileptic network architecture \cite{jerger2005multivariate, schindler2006assessing, schindler2008evolving, kramer2010coalescence, jiruska2012synchronization}, however the neurophysiologic roles of fine-grained areas of the network remain elusive. This thesis is designed to address \textbf{a gap} between dynamic network formalisms and the network drivers of meso-scale neurophysiologic events in neocortical epilepsy. Preliminary work from our laboratory suggests that meso-scale interactions that define the epileptic network can be extracted \cite{khambhati2014dynamic}.

There are several reasons why fulfilling our specific aims have great significance:
\begin{enumerate}
    \item \textit{Antiepileptic devices for precise modulation of epileptic networks:} Cortical targets for therapy are currently selected based on pathologic tissue associated with the earliest change from background in ECoG signal leading into a seizure \cite{litt2001epileptic}. A major shortcoming of this approach is that it discards spatial relationships that demonstrate how the seizure focus interacts with the surrounding epileptogenic network. \textit{I believe that identifying dynamic functional relationships within the epileptic network is vital, because epileptogenic regions cause symptoms not only through their own dysfunction, but also their ability to recruit and disrupt normal brain regions} \cite{kutsy1999ictal, spencer2002neural}.
    
    \item \textit{Reduce duration of invasive monitoring:} Invasive monitoring with ECoG sensors that lie on the cortical surface presents risks of infection. Despite attempts to automate and expedite target localization by selecting cortical areas with high emission rates of inter-ictal epileptiform events without having to study actual seizure semiology, there is no clear indication that these regions are suitable targets for therapy \comment{[needs refs]}. However, studies have shown that epileptiform events follow spatial trajectories similar to seizure spread \cite{alarcon1997origin, lai2007cortical, wilke2009identification}. \textit{I believe that latent network abnormalities extracted during inter-ictal periods will help determine targets for therapy.}
\end{enumerate}
