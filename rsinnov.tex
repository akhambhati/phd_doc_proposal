\section{Innovation}
Pioneering work has motivated a novel perspective that epilepsy is a disease of neuronal circuitry through which spreading pathologic activity disrupts neural function (i.e. rhythmic motor activity, altered cognition, or abnormal sensation) \cite{kutsy1999ictal, spencer2002neural}. Since, investigators have focused on the spatial distributions of epileptic activity using multiscale neural signals in ECoG and sub-millimeter $\mu$ECoG to more accurately localize where seizures start and how their pathologic activity affects distributed cortical structures \cite{worrell2008high-frequency, schevon2009spatial, stead2010microseizures, viventi2011flexible, feldt_muldoon2013spatially, weiss2013ictal}. More formally, investigators seek the functional organization of a complex epileptic network \cite{spencer2002neural, kramer2012epilepsy, lehnertz2014evolving}.
~\\
~\\
\bflin{Aim 1:} \textbf{Quantitatively identify topological structure in the epileptic network as seizures begin and evolve.}
For this aim I will \textbf{innovate} novel methods to study how network topology reorganizes over time and relate specific functional connections to network dynamics. Prior fMRI studies have explored gross whole-brain network reorganization during task states \cite{bassett2006adaptive, bassett2011dynamic}, however these approaches have neither been adapted to study meso-scale network structure nor been used to relate network phenomena to neurophysiologic events. The developed algorithms will need to assess network geometry to separate seizure-generating from surrounding epileptogenic network components and track functional connections \cite{holme2012temporal} within and between each set of components.
~\\
~\\
\bflin{Aim 2:} \textbf{Localize inter-ictal topological structure of the epileptic network compared to the seizure-generating network.}
Functional communication pathways of a network can be interpreted as indicators of network regions that co-activate under some particular set of brain dynamics within a window of time. Because these functional relationships arise due to hard-wired structural connectivity, through either gray or white matter, pathologic structure in the epileptic network are believed to be discernible not just during seizures but also inter-ictally. For this aim I will \textbf{innovate} network decomposition algorithms that can relate network topology between seizures and the inter-ictal periods that precede them. These decomposition approaches will need to dissociate healthy inter-ictal sub-network topologies from their pathologic counterparts.
~\\
~\\
\bflin{Aim 3:} \textbf{Compare localization of ictal and inter-ictal topological structure to the clinically defined seizure-onset zone and to clinical outcome.}
A variety of factors affect the efficacy by which clinicians delineate and impact the epileptic network. From an anatomical perspective the presence of a lesion on brain imaging immediately improves the chances of seizure freedom through surgical resection \cite{french2007refractory}. From a functional perspective seizures may either remain focal or secondarily spread, and implicate a much larger area of the epileptogenic network, affecting the target size for therapy. There is a lacking quantitative description to discriminate actual therapeutic targets from normal functioning cortex. An objective, automated, network-based approach for mapping the epileptic network provides this description alongside a novel and \textbf{innovative} way to compare network neurophysiology within and between clinical cohorts of different etiology. I will develop new imaging techniques to retrospectively compare overlap between suggested epileptic network targets and true resection targets. Statistical modelling can subsequently relate, and potentially predict, surgical outcome based on abnormal network components.

