% SPECIFIC AIMS -- PLACEHOLDER
Localization-related epilepsy is traditionally characterized by seizures that arise from one or more abnormal islands of cortical tissue, or `foci', in the neocortex or mesial temporal structures. In more severe cases, seizures with focal onset secondarily generalize, as pathologic activity spreads across the brain. For these patients the only treatment options are implantable neuromodulation devices, or more traditionally resective surgery. In surgical cases without discrete lesions on brain MRI associated with seizure onset (‘foci’) only $\approx$40\% remain seizure-free post-surgery. This modest outcome has lead investigators to further explore spatial distributions of epileptic activity to more accurately localize where seizures start and how their pathologic activity spreads. These approaches have spurred a paradigm shift from localizing epileptic ‘foci’ towards mapping the epileptic network and structural and functional architecture within it.

While seizures have recently been characterized by complex temporal dynamics involving synchronous and asynchronous processes, the structural and functional connections responsible for these dynamics remain elusive. Characterizing dynamic functional connectivity is critical to understand how the epileptic network reorganizes through seizures that (i) start as pathologic activity in a focal domain, (ii) may evolve beyond the epileptogenic cortex, disrupting function in normal brain regions, and (iii) self-terminate. Equally important is why sub-clinical epileptiform events with temporal structure similar to the seizure onset period do not evolve into full-blown clinical seizures. The disparity in how epileptic networks transition following sub-clinical events and onset of clinical seizures can elucidate which functional connections cause seizures to progress. Moreover, epileptic events that manifest as seizures potentially propagate through white-matter fibers that define a core structural component of the epileptic network. These ideas lead to the following three overarching themes of this thesis proposal:
\begin{enumerate}[topsep=1ex, itemsep=0pt]
    \item How does the epileptic network reorganize during seizures?
    \item Why do sub-clinical epileptiform events not progress into clinical seizures?
    \item What role does structural architecture play in defining the epileptic network?
\end{enumerate}

%%% AIM 1
\hangindent=6ex
\noindent
\bflin{Aim 1:} \textbf{Characterize dynamic functional architecture in human epileptic networks during seizures.}

\hangindent=6ex
\textbf{Hypothesis:} Functional connectivity within the epileptic network reconfigures as seizures evolve through onset, propagation and termination.

\hangindent=6ex
\textbf{Challenge:} To define the epileptic network from intracranial neural sensors and to classify and track functional connections within such network.

\hangindent=6ex
\textbf{Approach:} I will design algorithms to assess spatial and temporal structure as networks reconfigure through different states, augmenting existing dynamic network analysis methods.

\hangindent=6ex
\textbf{Impact:} These algorithms will quantify dynamic spatial interactions within the epileptic network during events of interest. These methods will lead to novel research and clinical tools to dissect and visualize epileptic network connectivity.


%%% AIM 2
\hangindent=\parindent
\hangafter=1
\noindent
\bflin{Aim 2:} \textbf{Relate functional architecture within the epileptic network during Isolate functional connections during Distinguish dynamic functional connectivity between transitions into and out of sub-clinical epileptiform events and transitions through seizure phases.}

\hangindent=6ex
\textbf{Hypothesis:} Critical functional relationships prevent sub-clinical events from evolving into clinical seizures.

\hangindent=6ex
\textbf{Challenge:} To group sub-clinical events with seizures of similar onset periods, and determine which characteristics of functional relationships to compare between the two types of events.

\hangindent=6ex
\textbf{Approach:} I will adapt approaches developed in \bflin{Aim 1} to compare network states during sub-clinical events to network states of seizures.

\hangindent=6ex
\textbf{Impact:} These experiments will identify specific functional connections that impact seizure evolution and provide insight on neurophysiologic mechanisms during transitions of sub-clinical events and clinical seizures.


%%% AIM 3
\hangindent=\parindent
\hangafter=1
\noindent
\bflin{Aim 3:} \textbf{Control the epileptic network away from seizure evolution states.}

\hangindent=6ex
\textbf{Hypothesis:} Dynamic functional connectivity can be driven to epileptic network states that promote seizure control.

\hangindent=6ex
\textbf{Challenge:} To identify controllability of the epileptic network from a pathologic state to a healthy state.

\hangindent=6ex
\textbf{Approach:} I will tailor existing network control paradigms to determine the controllability of epileptic networks. I will also propose an input paradigm to drive the epileptic network away from pathologic states.

\hangindent=6ex
\textbf{Impact:} These methods will help innovate new control paradigms to affect epileptic, and more generally brain, networks. They will augment existing neuromodulation methods used to prevent seizures.
