\section{Approach}
This thesis will answer several important questions, in the engineering, science and clinical realms, with regards to diagnosing functional architecture and identifying surgical targets in localization-related epileptic networks.

\subsection{Collaborations}
For guidance in engineering methodology I will leverage collaborations with Danielle Bassett, an expert in complex networks at the University of Pennsylvania who studies structural and functional brain networks, and Gershon Buchsbaum, an expert in signal processing at the University of Pennsylvania who can advise on signal dynamics. For guidance in clinical neuroscience I will collaborate with Timothy Lucas, a neurosurgeon at the University of Pennsylvania and expert in targeted surgery and neuromodulation.

\subsection{Patients}
All patients who have undergone surgical treatment for localization-related medically refractory epilepsy believed to be of neocortical origin at the Hospital of the University of Pennsylvania (HUP) and Mayo Clinic and have consented to their de-identified data being published on the online International Epilepsy Electrophysiology Portal (IEEG Portal) will be included in this study \cite{wagenaar2013multimodal}. Patients will be continuously recruited over the course of the study; thus far the study population includes 12 patients from HUP and 30 patients from Mayo. Clinically-determined seizure onset zone and earliest electrographic change for seizure initiation will be collected for all patients. Post-surgical outcome and resection zone will be collected to evaluate accuracy and specificity of methods developed in this thesis.

\subsection{Electrode Placement and Data Collection}
Intracranial electrodes will be placed according to standard clinical protocols decided by committee at epilepsy surgery conference based on preoperative imaging, clinical semiology, and structural imaging (PET, MRI). Electrodes are manufactured by either Ad-Tech Inc. or PMT Corp. and all non-artifact surface electrodes will be considered in formulating a working definition of a patient's epileptic network. Surface electrode configurations consist of linear and two-dimensional arrays with contacts 2.3 mm in diameter and spaced 10 mm apart. Since only epilepsies of neocortical origin are considered depth electrode information will be discarded. Ideal study patients should have at least one grid electrode (8x8 configuration) covering the presumed seizure onset area and the surrounding epileptogenic region.

ECoG signals are continuously recorded and digitized at a minimum 500 Hz sampling rate and converted off-line to the \textit{multiscale electrophysiology format} (MEF) \cite{brinkmann2009large-scale}.
