% SPECIFIC AIMS -- PLACEHOLDER
Localization-related epilepsy accounts for $\sim$80\% of drug-resistant cases and is traditionally characterized by seizures that arise from one or more abnormal islands of cortical tissue, or `foci', in the neocortex ($\sim$35\%) or mesial temporal structures ($\sim$65\%). For these patients the only treatment options are implantable neuromodulation devices, or more traditionally resective surgery. Modest outcomes associated with resective surgery has lead investigators to further explore spatial distributions of epileptic activity to more accurately localize where seizures start and how their pathologic activity spreads. These approaches have spurred a paradigm shift from localizing epileptic ‘foci’ towards mapping epileptic network architecture.

While seizures are often described by their complex temporal dynamics, the functional connections responsible for spatial interactions within the epileptogenic network remain elusive. Mapping network architecture can broadly impact theoretical understanding of seizure mechanisms and clinical therapy for managing seizures. Specifically, this thesis asks the following three overarching questions regarding neocortical epileptic network structure and function:
\begin{enumerate}[topsep=1ex, itemsep=0pt]
    \item Which functional connections play a role in the generation, propagation, and self-termination of seizures?
    \item How is abnormal functional connectivity expressed inter-ictally?
    \item Which components of the epileptic network are promising targets for therapy?
\end{enumerate}

~\\
%%% AIM 1
\hangindent=\parindent
\hangafter=1
\noindent
\bflin{Aim 1:} \textbf{Identify topological abnormalities in the epileptic network as seizures begin and evolve.}

\hangindent=6ex
\textbf{Hypothesis:} Abnormalities in epileptic network functional organization dissociate a seizure-generating network and surrounding epileptogenic network.

\hangindent=6ex
\textbf{Challenge:} To define the epileptic network from intracranial neural sensors and to classify and track functional connections within such network.

\hangindent=6ex
\textbf{Approach:} I will design algorithms to assess spatial and temporal structure as networks reconfigure through different states, augmenting existing dynamic network analysis methods.

\hangindent=6ex
\textbf{Impact:} These methods will elucidate how neocortical seizures begin and what changes in connectivity enable seizure activity to evolve and terminate.

~\\
%%% AIM 2
\hangindent=\parindent
\hangafter=1
\noindent
\bflin{Aim 2:} \textbf{Localize seizure-generating network and differentiate from surrounding epileptogenic network through inter-ictal topological abnormalities.}

\hangindent=6ex
\textbf{Hypothesis:} Fundamental topological abnormalities of the epileptic network can map the seizure-generating network during inter-ictal periods.

\hangindent=6ex
\textbf{Challenge:} To isolate inter-ictal functional domains of the epileptic network and relate these domains to network organization during seizures.

\hangindent=6ex
\textbf{Approach:} I will employ network decomposition methods and adapt approaches developed in \bflin{Aim 1} to compare functional organization between inter-ictal and ictal periods.

\hangindent=6ex
\textbf{Impact:} Mapping the seizure generating network and surrounding epileptogenic network can help better target regions of interest for therapy and reduce clinical monitoring time of invasive procedures.

~\\
%%% AIM 3
\hangindent=\parindent
\hangafter=1
\noindent
\bflin{Aim 3:} \textbf{Evaluate localization accuracy and specificity of abnormal epileptic network components in patients.}

\hangindent=6ex
\textbf{Hypothesis:} Degree of overlap between resection zone and abnormal epileptic network components positively correlates with patient outcome.

\hangindent=6ex
\textbf{Challenge:} To statistically quantify overlap between the resection zone and network components.

\hangindent=6ex
\textbf{Approach:} I will determine resection zone from post-operative imaging and employ bootstrap statistical methods to compare overlap between localized network structures (from \bflin{Aims 1 and 2}) and resection zone.

\hangindent=6ex
\textbf{Impact:} This aim will evaluate objective algorithms that decompose functional networks and localize abnormalities for mapping surgical targets in neocortical epilepsy.
