% SPECIFIC AIMS -- PLACEHOLDER



%%% AIM 1
\hangindent=6ex
\noindent
\bflin{Aim 1:} \textbf{Characterize dynamic functional interactions in naturally-occurring epileptic networks.}

\hangindent=6ex
\textbf{Hypothesis:} Functional connections within the epileptic network reconfigure during epileptiform activity.

\hangindent=6ex
\textbf{Challenge:} To determine dynamic functional connectivity from intracranial recordings and classify and track this connectivity in order to understand time-dependent topological properties of epileptic networks.

\hangindent=6ex
\textbf{Approach:} I will develop connectivity metrics to compare signals from intracranial neural sensors. I will also design algorithms to relate network configurations over time, augmenting existing dynamic network analysis methods.

\hangindent=6ex
\textbf{Impact:} These algorithms will help quantify the spatial involvement of distributed cortical structures during epileptiform events. Relating geometric space of sensors to epileptic network topology will also help determine which circuits to target with implantable neuromodulation devices.


%%% AIM 2
\hangindent=\parindent
\hangafter=1
\noindent
\bflin{Aim 2:} \textbf{Determine how intraoperative functional connectivity relates to extraoperative functional connectivity during epileptiform events.}

\hangindent=6ex
\textbf{Hypothesis:} Functional connectivity based on intraoperative intracranial recordings can help map the epileptic network.

\hangindent=6ex
\textbf{Challenge:} To investigate effects of anesthesia on functional connectivity and retrospectively characterize connections in intraoperative functional networks involved in epileptiform events.

\hangindent=6ex
\textbf{Approach:} I will collect high-bandwidth intracranial recordings in the operating room under anesthesia types known to enhance epileptiform activity. I will adapt approaches developed in \bflin{Aim 1} to characterize functional connectivity in intraoperative epileptic networks and compare to extraoperative network configurations.

\hangindent=6ex
\textbf{Impact:} These methods will yield relationships between connections present under extraoperative and intraoperative conditions. They can also inform online network classification methods to immediately map the epileptic network intraoperatively.


%%% AIM 3
\hangindent=\parindent
\hangafter=1
\noindent
\bflin{Aim 3:} \textbf{Localize and relate structural white-matter architecture underlying functional communication in epileptic networks.}

\hangindent=6ex
\textbf{Hypothesis:} Structural connectivity is predictive of spatial spread of pathologic epileptiform activity.

\hangindent=6ex
\textbf{Challenge:} To establish white-matter fiber connectivity between regions of interest within the epileptic network and determine which tracts propagate epileptic activity.

\hangindent=6ex
\textbf{Approach:} I will engineer imaging techniques to reconstruct and track fiber pathways that stem from intracranial sensor locations. I will then expand statistical algorithms developed in \bflin{Aim 1} to constrain functional relationships based on structural connections.

\hangindent=6ex
\textbf{Impact:} These methods will help trace white-matter fiber pathways that are complicit in functional dynamics of epileptic events. They will provide new clinical interventional approaches that prevent epileptiform activity from propagating to cortical structures critical to eloquent function.
