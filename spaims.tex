% SPECIFIC AIMS -- PLACEHOLDER



%%% AIM 1
\hangindent=6ex
\noindent
\bflin{Aim 1:} \textbf{Characterize the dynamic functional interactions in naturally-occurring epileptic networks.}

\hangindent=6ex
\textbf{Hypothesis:} Functional connections within the epileptic network reconfigure during epileptiform activity.

\hangindent=6ex
\textbf{Challenge:} To determine dynamic functional connectivity from intracranial recordings and classify and track this connectivity in order to understand time-dependent topological properties of epileptic networks.

\hangindent=6ex
\textbf{Approach:} I will develop connectivity metrics to compare signals from intracranial neural sensors. I will also design algorithms to relate network configurations over time, augmenting existing dynamic network analysis methods.

\hangindent=6ex
\textbf{Impact:} These algorithms will help quantify the spatial involvement of distributed cortical structures during epileptiform events. Relating geometric space of sensors to epileptic network topology will also help determine which circuits to target with implantable neuromodulation devices.


%%% AIM 2
\hangindent=\parindent
\hangafter=1
\noindent
\bflin{Aim 2:} \textbf{Investigate intraoperative reminiscence of functional connectivity in epileptic networks.}

\hangindent=6ex
\textbf{Hypothesis:} Functional connectivity based on intraoperative electrophysiology will relate to network configurations exhibited during naturally-occurring seizures.

\hangindent=6ex
\textbf{Challenge:} To understand effects of anesthesia on functional connectivity in epileptic networks and retrospectively extract connections involved in seizures.

\hangindent=6ex
\textbf{Approach:} I will record high-bandwidth electrophysiology in the operating room under anesthesia types known to enhance particular aspects of signal spectrum. I will characterize functional connectivity in intraoperative epileptic networks and compare network configurations to those during seizures using approaches developed in \bflin{Aim 1}.

\hangindent=6ex
\textbf{Impact:} These methods will yield relationships between connections present under awake and intraoperative conditions. They will also inform online connectivity classification methods to immediately determine network involvement in seizures intraoperatively.


%%% AIM 3
\hangindent=\parindent
\hangafter=1
\noindent
\bflin{Aim 3:} \textbf{Localize and relate structural white-matter architecture underlying functional communication in epileptic networks.}

\hangindent=6ex
\textbf{Hypothesis:} Structural connectivity is predictive of spatial spread of pathologic epileptiform activity.

\hangindent=6ex
\textbf{Challenge:} To establish white-matter fiber connectivity between regions of interest within the epileptic network and determine which tracts propagate epileptic activity.

\hangindent=6ex
\textbf{Approach:} I will engineer imaging techniques to reconstruct and track fiber pathways that stem from intracranial sensor locations. I will then expand statistical algorithms developed in \bflin{Aim 1} to constrain functional relationships based on structural connections.

\hangindent=6ex
\textbf{Impact:} These methods will help trace white-matter fiber pathways that are complicit in functional dynamics of epileptic events. They will provide new clinical interventional approaches that prevent epileptiform activity from propagating to cortical structures critical to eloquent function.
