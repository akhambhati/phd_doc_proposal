% THE ABSTRACT GOES HERE
Mapping the functional architecture of the epileptic network is a promising method to objectively localize cortical targets for therapy in cases of neocortical refractory epilepsy, where post-surgical seizure freedom is low when cortical structures responsible for generating seizures are difficult to delineate. The effectiveness of traditional methods that attempt to detect and predict functional abnormalities has been limited, in part because of their focus on ad-hoc features from individual cortical sensors that have little neurophysiologic interpretability. Recent work from our group highlights network-based approaches to extract specific spatial interactions, such as synchronization and desynchronization, between neural populations complicit in seizure dynamics. This thesis proposal will contribute a basic science understanding of dynamic network mechanisms that drive neocortical seizure activity and will assess the associated analytical tools and algorithms for clinical translation.
