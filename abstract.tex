% THE ABSTRACT GOES HERE
Mapping functional architecture in the epileptic network is promising for objectively localizing cortical targets for therapy in cases of neocortical refractory epilepsy, where post-surgical seizure freedom is unfavorable when cortical structures responsible for generating seizures are difficult to delineate. The effectiveness of traditional methods that attempt to detect and predict functional topological structure has been limited, in part because of their focus on ad-hoc features from individual cortical sensors that have little neurophysiologic interpretability. A recent shift has researchers considering how distributed cortical structures interact through synchronization and desynchronization mechanisms to drive seizure dynamics. However, this body of work concentrates on gross network architecture. Preliminary work from our group highlights network-based approaches to extract and dynamically track specific connections in the epileptic network during seizures. These methods are vital for relating pathologic connectivity back to candidate cortical structures for targeted therapy. This thesis proposal will contribute a basic science understanding of dynamic network mechanisms that drive neocortical seizure activity and will assess functional network-based tools and algorithms to objectively detect and potentially predict network structures of interest for clinical translation.
